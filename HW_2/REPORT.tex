
\def \Authora {محمدحسین حسنی - 99521199}
\def \Authorb {ملیکا محمدی فخار - ‫‪99522086‬‬}

\begin{center}
\vspace{.1cm}
\vspace{.1cm}
{\bf \Authora }  \\
{\bf \Authorb }  \\
\end{center}
\hspace{\fill} 
\vspace{0.10cm}

\clearpage


\section{مقایسه بین FDMA و TDMA تفاوت ها، مزایا و معایب نسبت به همدیگر}

FDMA و TDMA دو روش اصلی برای مدیریت و استفاده از فضای فرکانس یک طیف رادیویی هستند. FDMA بر اساس تقسیم ثابت فرکانس‌ها به باندهای جداگانه عمل می‌کند، در حالی که TDMA از شکاف‌های زمانی برای انتقال اطلاعات استفاده می‌کند.

FDMA سادگی در پیاده‌سازی دارد و هر کاربر یا دستگاه از یک باند فرکانسی خاص برای انتقال داده‌ها استفاده می‌کند. این روش مقاومت قابل توجهی در برابر تداخل فرکانسی دارد، اما ناکارآمد می‌شود در صورت نیاز به نرخ داده‌های متغیر یا در مواجهه با تعداد زیادی کاربر که باعث محدودیت‌های ارتباطی می‌شود.

TDMA به کاربران شکاف‌های زمانی مخصوصی اختصاص می‌دهد که در آن زمان می‌توانند اطلاعات خود را ارسال کنند. این روش امکان تخصیص پویا و مناسب به نیازهای نرخ داده‌های مختلف را دارد و معمولاً با تداخل کمتری روبروست، اما نیاز به همگام‌سازی دقیق بین کاربران و سربار بیشتری در مواجهه با تعداد زیادی کاربر دارد.

در مقایسه، TDMA اغلب بهترین عملکرد را از نظر بهره‌وری فرکانس و تطبیق با نیازهای مختلف کاربران ارائه می‌دهد. در حالی که FDMA سادگی در اجرا و مقاومت در برابر تداخل فرکانسی دارد، اما به نسبت TDMA در مواجهه با نیازهای نرخ داده‌های متغیر و تعداد زیاد کاربران کمتر کارآمد است.

\section{مقایسه بین TDD و FDD تفاوت ها، مزایا و معایب نسبت به همدیگر}

TDD و FDD دو روش اصلی در شبکه‌های ارتباطی برای جداسازی ارسال و دریافت داده‌ها می‌باشند. در FDD، ارسال و دریافت اطلاعات از طریق باندهای فرکانسی جداگانه صورت می‌گیرد و در TDD، ارسال اطلاعات در زمان‌های متفاوتی از یک باند فرکانسی انجام می‌شود.

FDD به کاربران امکان انتقال و دریافت همزمان داده‌ها را می‌دهد؛ به این معنا که از باندهای فرکانسی مختلف برای ارتباط در لینک فراسو (از کاربر به ایستگاه پایه) و لینک فروسو (از ایستگاه پایه به کاربر) استفاده می‌کند. این ویژگی باعث عملکرد بهتر و سازگاری مناسب‌تر در سیستم می‌شود. اما نیاز به باندهای فرکانسی جداگانه باعث افزایش نیاز به طیف فرکانسی و پیچیدگی در پیاده‌سازی می‌شود.

در TDD، از یک باند فرکانسی واحد برای انتقال هر دو جهت (ارسال و دریافت) استفاده می‌شود. این روش از شکاف‌های زمانی مختلف برای انتقال داده‌ها استفاده می‌کند و به کاربر امکان تخصیص انعطاف‌پذیرتر منابع بر اساس نیازهای ترافیکی را می‌دهد. TDD می‌تواند از نظر بهره‌وری طیف، به دلیل استفاده پویای زمان، موثرتر باشد. اما مدیریت همگام‌سازی بین ارسال‌های uplink و downlink و مواجهه با نیازهای نامتقارن ترافیک ممکن است چالش‌هایی ایجاد کند.

\section{چرا فرکانس پیوند فروسو از فراسو در FDD بالاتر است؟ ایا این اتفاق در مخابرات ماهواره‌ای نیز به همین صورت است؟}

در سیستم‌های FDD، فرکانس پایین‌تر برای ارسال داده‌ها از کاربر به مرکز ارتباطی (uplink) و فرکانس بالاتر برای دریافت داده‌ها از مرکز به کاربر (downlink) استفاده می‌شود. این جداسازی فرکانسی در FDD به منظور کاهش تداخل بین دو جهت ارتباطی و بهبود عملکرد سیستم مورد استفاده قرار می‌گیرد.

این انتخاب فرکانس‌های بالاتر برای downlink به دو دلیل اصلی برمی‌گردد: کاهش تداخل و نیازهای نظارتی. فرکانس‌های بالاتر معمولاً در مقایسه با فرکانس‌های پایین‌تر، کمتر تحت تأثیر نوسانات جوی و دارای دامنه کوتاه‌تری هستند. این انتخاب باعث کاهش تداخل بین ارسال‌های uplink و downlink می‌شود و اجازه می‌دهد که دو سیگنال به صورت موازی و بدون اشکال در یک طیف فرکانسی کنار هم کار کنند.

در مخابرات ماهواره‌ای، اصل جداسازی فرکانس‌ها نیز به دلایل مشابه به کار می‌رود. فرکانس uplink (ارسال از زمین به ماهواره) معمولاً در باندهای فرکانس پایین‌تر قرار دارد (مانند باندهای C و Ku)، در حالی که فرکانس downlink (ارسال از ماهواره به زمین) در باندهای فرکانس بالاتر (مانند باندهای Ka و V) قرار می‌گیرد. این جداسازی فرکانسی کمک می‌کند تا تداخل بین سیگنال‌های ارسالی از ماهواره و سیگنال‌های ارسال شده از زمین به ماهواره به حداقل برسد.

انتخاب باندهای فرکانسی در ارتباطات ماهواره‌ای نیز تحت تأثیر عواملی مانند تغییرات هواشناسی، محدودیت‌های موجودیت طیف فرکانسی و ویژگی‌های انتشار مختلف باندهای فرکانسی در محیط فضایی قرار دارد. این انتخاب‌ها با هدف بهینه‌سازی عملکرد و کاهش تداخل در سیستم‌های مخابراتی ماهواره‌ای صورت می‌گیرد.


\section{کدهای متعامد چگونه تولید می شوند؟ نقش چندجمله های مولد در این میان چیست؟}

کدهای متعامد در سیستم‌های CDMA با استفاده از الگوهای ریاضی مانند کدهای والش یا هم‌چنین کدهای هادامارد ایجاد می‌شوند. کدهای والش، که از ماتریس هادامارد ساخته می‌شوند، یک مجموعه از کدهای متعامد هستند که در شبکه‌های CDMA استفاده می‌شوند. این کدها بر پایهٔ ماتریس هادامارد ساخته می‌شوند که یک ماتریس مربع از اندازه $2^n$ است (در اینجا 'n' یک عدد صحیح است). برای هر مقدار n، $2^n$ کد والش مختلف وجود دارد و این کدها از ردیف‌های ماتریس هادامارد تولید می‌شوند.

کدهای هادامارد نیز به صورت بازگشتی تولید می‌شوند و ویژگی مهم آن‌ها این است که سطرهای (یا ستون‌های) آن‌ها متعامد هستند. این خاصیت متعامد بودن در سیستم‌های CDMA بسیار حیاتی است؛ زیرا به چندین کاربر اجازه می‌دهد که باند فرکانسی مشابهی را بدون تداخل قابل توجه با یکدیگر به اشتراک بگذارند.

کدهای متعامد دارای ویژگی‌های مهمی مانند متعامد بودن، تعادل، استفاده از طیف کارآمد، مقاومت در برابر خطا و سادگی رمزگشایی هستند. این کدها برای تمایز بین سیگنال‌های مختلف ارسالی از کاربران، به گیرنده‌ها کمک می‌کنند.

چندجمله‌ای های مولد نیز در تولید دنباله‌های استفاده شده در سیستم‌های CDMA نقش مهمی دارند. آن‌ها نمایش‌دهنده‌هایی از توالی‌ها هستند که برای تولید دنباله‌های مورد استفاده در این سیستم‌ها به کار می‌روند. این چندجمله‌ای ها، ویژگی‌های مانند همبستگی و متقابل بین دنباله‌های تولید شده را تعیین می‌کنند. به عنوان مثال، در یک سیستم CDMA، هر کاربر دارای یک دنباله (کد) منحصر به فرد است که برای تمایز سیگنال‌های ارسالی از سایر کاربران استفاده می‌شود.

به طور خلاصه، کدهای متعامد و چندجمله‌ای های مولد در CDMA نقش مهمی در تولید دنباله‌هایی با خصوصیات مانند متعامد بودن، تعادل، و کارایی در محیط‌های با تداخل دارند. این ویژگی‌ها به سیستم‌های CDMA کمک می‌کنند تا بتوانند با اشتراک‌گذاری طیف فرکانسی یکسان، بین ارسال‌های کاربران مختلف تمایز ایجاد کنند.


\section{در مورد مزایا و معایب CDMA نسبت FDMA و TDMA چیست؟}

\textbf{مزایای CDMA:}

\begin{itemize}
    \item \textbf{افزایش ظرفیت:} CDMA به طور معمول ظرفیت بالاتری نسبت به FDMA و TDMA فراهم می‌کند که به چندین کاربر اجازه می‌دهد که به صورت همزمان در یک باند فرکانسی مشترک اطلاعات را انتقال دهند بدون تداخل.
    \item \textbf{کیفیت تماس بهتر:} CDMA کیفیت تماس و ظرفیت بهتری در حضور نویز و تداخلات فراهم می‌کند، زیرا از تکنولوژی پخش طیف استفاده می‌کند که می‌تواند تأثیر نویز و تداخل را کاهش دهد.
    \item \textbf{محدودیت نرم ظرفیت:} CDMA محدودیت سختی در تعداد کاربرانی که می‌تواند پشتیبانی کند ندارد، اما با افزودن بیشتر کاربران، سیستم ممکن است به تدریج در سرعت انتقال داده کاهش یابد.
    \item \textbf{امنیت بهبود یافته:} CDMA به دلیل طبیعت گسترده‌ی طیف اطلاعات، سطحی از امنیت ارائه می‌دهد.
\end{itemize}

\textbf{معایب CDMA:}

\begin{itemize}
    \item \textbf{پیاده‌سازی پیچیده:} سیستم‌های CDMA نسبت به سیستم‌های FDMA و TDMA پیچیده‌تر باشند و نیاز به تکنیک‌های پردازش سیگنال پیشرفته دارند.
    \item \textbf{مشکل نزدیک-دور:} CDMA ممکن است با مشکل "نزدیک-دور" روبه‌رو شود که نیاز به مکانیسم‌های کنترل توان دارد.
    \item \textbf{سازگاری محدود:} تکنولوژی CDMA به اندازه GSM به عنوان گزینه‌ای عمومی شناخته نشده است که می‌تواند تعداد کمتری از امکانات تعاملی و پیمایشی سیستم‌های مبتنی بر CDMA را محدود کند.
    \item \textbf{بازده طیف کمتر:} کارایی طیفی در برخی موارد ممکن است کمتر از سیستم‌های FDMA و TDMA طراحی شده باشد.
\end{itemize}


\section{در مورد تمایز بین CSMA، CD/CSMA و CA/CSMA تحقیق کنید.}

\subsection*{CSMA (چندین دسترسی با حسگر حامل)}

\begin{enumerate}
    \item \textbf{پروتکل اصلی}:
    \begin{itemize}
        \item CSMA یک پروتکل ابتدایی است که در آن یک دستگاه به حامل ارتباطی (حامل) گوش می‌دهد قبل از ارسال.
    \end{itemize}
    
    \item \textbf{مدیریت تصادف}:
    \begin{itemize}
        \item اگر حامل مشغول باشد، دستگاه منتظر می‌ماند تا خالی شود و سپس ارسال می‌کند. با این حال، CSMA تصادف را شناسایی نمی‌کند. اگر دو دستگاه همزمان ارسال کنند، تصادف ممکن است رخ دهد و هر دو دستگاه ممکن است نیاز به ارسال مجدد داشته باشند.
    \end{itemize}
    
    \item \textbf{استفاده در اترنت}:
    \begin{itemize}
        \item CSMA معمولاً در شبکه‌های اترنت استفاده می‌شود که دستگاه‌ها برای دسترسی به حامل مشترک رقابت می‌کنند.
    \end{itemize}
\end{enumerate}

\subsection*{CSMA/CD (چندین دسترسی با حسگر حامل و شناسایی تصادف)}

\begin{enumerate}
    \item \textbf{شناسایی تصادف}:
    \begin{itemize}
        \item CSMA/CD با افزودن قابلیت‌های شناسایی تصادف CSMA را گسترش می‌دهد. دستگاه‌ها می‌توانند تصادفات را در حین ارسال شناسایی کنند و ارسال را متوقف کنند تا از تصادفات بیشتر جلوگیری کنند.
    \end{itemize}
    
    \item \textbf{مدیریت تصادف}:
    \begin{itemize}
        \item اگر تصادفی شناسایی شود، دستگاه‌های معنی‌دار فوراً ارسال را متوقف کرده و دوره انتظار قبل از تلاش برای ارسال مجدد را آغاز می‌کنند.
    \end{itemize}
    
    \item \textbf{استفاده در اترنت (سنتی)}:
    \begin{itemize}
        \item CSMA/CD تاریخچه استفاده در شبکه‌های اترنت سنتی را دارد. با انتشار گسترده اترنت تمام دوطرفه و سوئیچ‌ها، نیاز به شناسایی تصادف کاهش یافته و شبکه‌های اترنت مدرن معمولاً فقط از CSMA/CD برای پشتیبانی از سابق استفاده می‌کنند.
    \end{itemize}
\end{enumerate}

\subsection*{CSMA/CA (چندین دسترسی با حسگر حامل و اجتناب از تصادف)}

\begin{enumerate}
    \item \textbf{اجتناب از تصادف}:
    \begin{itemize}
        \item CSMA/CA به جای تشخیص تصادف بر روی اجتناب از تصادف تمرکز دارد. این به طور معمول در شبکه‌های بی‌سیم استفاده می‌شود.
    \end{itemize}
    
    \item \textbf{فریم‌های درخواست ارسال (RTS) و تأیید ارسال (CTS)}:
    \begin{itemize}
        \item در CSMA/CA، دستگاه‌ها از یک فرآیند با استفاده از فریم‌های درخواست ارسال (RTS) و تأیید ارسال (CTS) استفاده می‌کنند. قبل از ارسال داده، یک دستگاه یک فریم RTS به مقصد ارسال می‌کند تا اجازه ارسال را درخواست کند. اگر مقصد در دسترس بوده و اجازه دهد، با یک فریم CTS پاسخ می‌دهد و انتقال داده واقعی انجام می‌شود.
    \end{itemize}
    
    \item \textbf{استفاده در شبکه‌های وای-فای}:
    \begin{itemize}
        \item CSMA/CA در شبکه‌های وای-فای برای مدیریت دسترسی به حامل بی‌سیم استفاده می‌شود. این به جلوگیری از تصادف با کنترل انتقال دستگاه‌ها کمک می‌کند.
    \end{itemize}
    
    \item \textbf{مکانیسم دوره انتظار}:
    \begin{itemize}
        \item CSMA/CA همچنین یک مکانیسم دوره انتظار مشابه CSMA/CD را شامل می‌شود. اگر یک انتقال تأیید نشود، دستگاه منتظر یک دوره تصادفی می‌ماند قبل از تلاش برای ارسال مجدد.
    \end{itemize}
\end{enumerate}

در خلاصه، CSMA یک پروتکل ابتدایی بدون شناسایی تصادف است، CSMA/CD شناسایی تصادف را اضافه می‌کند و در گذشته در اترنت سنتی استفاده می‌شد، و CSMA/CA بر روی اجتناب از تصادف تمرکز دارد و به طور معمول در شبکه‌های بی‌سیم مانند وای-فای استفاده می‌شود. انتخاب پروتکل بستگی به نیازها و ویژگی‌های خاص محیط شبکه دارد.


